Die Langmuir-Isotherme ist das einfachste Sorptionsmodell, das physikalische Grundlagen besitzt. 
Es geht von folgenden Annahmen aus:


\begin{itemize}
\item Adsorption nur in einer einzelnen molekularen Schicht
\item alle Sorptionsplaetze sind gleichwertig
\item die Oberflaeche ist gleichfoermig 
\item keine Wechselwirkungen zwischen benachbarten Sorptionsplaetzen und den adsorbierten Teilchen
\item noch ein neuer punkt. toll oder 
\item nochmals so ein toller neuer punkt
\end{itemize}


Die Langmuir-Isotherme kann eine maximale Beladung der Sorptionsoberflaechen abbilden und ist damit Ausgangsbasis für weitere Adsorptionsmodelle (Gleichung 1):


\begin{equation}
q=q_{max}*\frac{K*[Si]}{1+K*[Si]}
\end{equation}


\begin{description}
\item[ ] $q$ = adsorbierte Menge pro Gramm Gibbsit
\item[ ] $q_{max}$ = maximal adsorbierte Menge pro Gramm Gibbsit
\item[ ]$K$ = Langmuir Koeffizient	
\item[ ]$[Si]$ = Si concentration
\end{description}
